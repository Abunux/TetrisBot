\chapter{Implémentation du moteur de jeu}

\section{Structures de données}
Le moteur de jeu est construit autour de trois classes :
\begin{itemize}
	\item \pyth{Tetramino} : les blocs 
	\item \pyth{Board} : la grille de jeu
	\item \pyth{TetrisEngine} : le moteur de jeu qui fait le lien entre les deux classes précédentes
\end{itemize} 

\section{La classe Tetramino}
La classe \pyth{Tetramino} est responsable de le gestion des blocs et de leurs rotations. Elle est implémentée dans le fichier \pyth{tetramino.py}.

Un bloc est défini par :
\begin{itemize}
	\item Un \pyth{id} qui permet d'identifier son type
	\item Un glyphe de base qui représente la pièce sans rotation dans une matrice carrée. Chaque case occupée par le bloc est codée par son \pyth{id} et les cases vides par 0.
	\item Le nombre de rotations (par exemple le O n'a qu'une seul rotation, le I en a deux et le T en a quatre)
\end{itemize}



\section{La classe Board}

\section{La classe TetrisEngine}

