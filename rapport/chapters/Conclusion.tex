\chapter{Conclusion}

Après plusieurs centaines d'heures de codage et de réflexion, 3000 lignes de code et quelques nuits blanches, nous avons obtenu des résultats tout à fait satisfaisants. Les agents sont plaisants à voir jouer et dépassent les capacités d'un joueur humain.

\medskip

Le fait d'implémenter plusieurs types d'agents ainsi que la nécessité d'avoir un moteur optimisé au maximum a été très stimulant.

\medskip

Il ne nous manque que l'implémentation du deep Q-Learning qui était peut-être un peu ambitieux vu le temps dont nous disposions. Nous avons quand même pu faire du simple Q-Learning et cela nous a permis de nous documenter et de comprendre le reinforcement learning.

\medskip

Il serait possible de faire évoluer le projet sur les points suivants :

\begin{itemize}
	\item Bien sûr, en implémentant le DQN
	\item En essayant de paralléliser les calculs dans les algorithmes génétiques
	\item En rajoutant une interface graphique, ce que nous avons délibérément choisi de ne pas faire ici
\end{itemize}

\medskip

Étant tous les deux enseignants, le côté pédagogique est important et ce projet, même s'il dépasse largement le niveau attendu d'un élève de lycée, pourrait servir de support pour travailler les points suivants :
\begin{itemize}
	\item Réflexion sur le codage d'une structure de données (pièces et grille)
	\item Organisation d'un projet sous forme de modules et programmation objet
	\item Algorithmique sur les tableaux (détermination des statistiques de la grille)
	\item Passage de fonctions en paramètres (callback)
	\item Gestion de l'affichage d'une grille
	\item Mise en place de stratégies simples pour les agents (par exemple le filtrage)
\end{itemize}

\vfill

\textit{\begin{flushright}
		To be continued...
\end{flushright}}