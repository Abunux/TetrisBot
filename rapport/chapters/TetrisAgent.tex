\chapter{Les agents}
Les agents sont à la base du projet dont le but est, justement, de les programmer.

\section{Généralités}
Un agent est essentiellement une classe qui a accès à un moteur de jeu (et donc à tous les paramètres de jeu) et qui implémente une méthode \pyth{getMove()} qui renvoie la commande du prochain coup à jouer. \\
C'est de leur responsabilité de lancer la partie.

Les fonctionnalités de base des agents sont implémentées dans la classe \pyth{Agent} dont les agents héritent tous. cette classe permet de :
\begin{itemize}
	\item Récupérer les paramètres de la grille après qu'un coup ait été joué via la méthode \pyth{getMoveStats(self, move)}. Cette méthode est utilisée dans la méthode \pyth{allMoveStats(self)} qui remplit un dictionnaire dont les clefs sont les différents placements possibles et les valeurs un dictionnaire content les différentes statistiques de jeu (nombre de lignes créées, nombre de trous, ...)
	\item Créer une commande pour un placement direct via la méthode \pyth{commandFromMove(self, move)}.
\end{itemize}

\section{Joueur humain en mode texte}
Pour tester les différentes fonctionnalités du moteur, nous avons implémenté un agent, \pyth{AgentHuman}, qui se contente de recevoir les différentes commandes à jouer via l'entrée standard.\\
Grâce à cet agent nous avons pu tester le déplacement et la rotation des pièces, la suppression des lignes, ...

\section{Agent aléatoire}
Ensuite le premier agent automatique qui joue de manière aléatoire. \\
\pyth{AgentRandom1} joue des coups aléatoires de type "aller à gauche", "tourner la pièce", ... à la manière d'un joueur humain.\\
\pyth{AgentRandom2}, quant à lui, place directement les pièces dans des colonnes et des rotations aléatoires.\\

Évidemment ces deux agents sont catastrophiques en terme de performances mais ne demandent qu'à être améliorés (ces sont les "hello world" des agents).

\section{Agent par filtrage}
Le premier agent un tant soit peu efficace.\\
La stratégie utilisée est de filtrer la liste des coups jouables successivement selon plusieurs critères :
\begin{itemize}
	\item D'abord il ne garde que les coups qui font le moins de trous
	\item Ensuite, parmi eux, il ne garde que ceux qui donnent une somme des hauteurs des colonnes de la structure minimale
	\item Puis, ceux qui minimisent le bumpiness
	\item Enfin ceux qui créent le plus de lignes 
\end{itemize} 

Cet agent joue plutôt bien pour une heuristique aussi simple mais il a tendance à créer des puits pour éviter de faire des trous.

\section{Agent par évaluation des coups}
\label{evaluation_des_coups}
%Cet agent est à la base de ce que nous ferons avec les algorithmes génétiques :\\
Pour chaque coup jouable notons $L$ le nombre de lignes, $H$ la somme des hauteurs des colonnes, $T$ le nombre de trous créés et $B$ le bumpiness, après que le coup a été joué.\\
La qualité d'un coup peut être évaluée par une fonction $$q(L,H,T,B)=a\times L - b\times H - c\times T - d\times B$$
%où $a$, $b$, $c$ et $d$ sont des paramètres positifs que nous pouvons supposer dans $[0~;~1[$ (en effet on pourrait, sans perte de généralité les diviser par $a+b+c+d$ pour s'y ramener si ce n'était pas le cas).\\
où $a$, $b$, $c$ et $d$ sont des paramètres positifs, et nous pouvons supposer que le vecteur $(a,b,c,d)$ est normé dans $\R^4$ (soit pour la norme $||\cdot||_1$, soit pour la norme $||\cdot||_2$).\\

Le coup à jouer est alors celui qui maximise cette fonction.\\

Tout le problème consiste donc à déterminer ces coefficients et c'est le but de l'optimisation par algorithme génétique.

