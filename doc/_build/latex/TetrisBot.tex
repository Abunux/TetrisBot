%% Generated by Sphinx.
\def\sphinxdocclass{report}
\documentclass[letterpaper,10pt,french]{sphinxmanual}
\ifdefined\pdfpxdimen
   \let\sphinxpxdimen\pdfpxdimen\else\newdimen\sphinxpxdimen
\fi \sphinxpxdimen=.75bp\relax

\usepackage[utf8]{inputenc}
\ifdefined\DeclareUnicodeCharacter
 \ifdefined\DeclareUnicodeCharacterAsOptional
  \DeclareUnicodeCharacter{"00A0}{\nobreakspace}
  \DeclareUnicodeCharacter{"2500}{\sphinxunichar{2500}}
  \DeclareUnicodeCharacter{"2502}{\sphinxunichar{2502}}
  \DeclareUnicodeCharacter{"2514}{\sphinxunichar{2514}}
  \DeclareUnicodeCharacter{"251C}{\sphinxunichar{251C}}
  \DeclareUnicodeCharacter{"2572}{\textbackslash}
 \else
  \DeclareUnicodeCharacter{00A0}{\nobreakspace}
  \DeclareUnicodeCharacter{2500}{\sphinxunichar{2500}}
  \DeclareUnicodeCharacter{2502}{\sphinxunichar{2502}}
  \DeclareUnicodeCharacter{2514}{\sphinxunichar{2514}}
  \DeclareUnicodeCharacter{251C}{\sphinxunichar{251C}}
  \DeclareUnicodeCharacter{2572}{\textbackslash}
 \fi
\fi
\usepackage{cmap}
\usepackage[T1]{fontenc}
\usepackage{amsmath,amssymb,amstext}
\usepackage{babel}
\usepackage{times}
\usepackage[Sonny]{fncychap}
\usepackage[dontkeepoldnames]{sphinx}

\usepackage{geometry}

% Include hyperref last.
\usepackage{hyperref}
% Fix anchor placement for figures with captions.
\usepackage{hypcap}% it must be loaded after hyperref.
% Set up styles of URL: it should be placed after hyperref.
\urlstyle{same}
\addto\captionsfrench{\renewcommand{\contentsname}{Table des matières}}

\addto\captionsfrench{\renewcommand{\figurename}{Fig.}}
\addto\captionsfrench{\renewcommand{\tablename}{Tableau}}
\addto\captionsfrench{\renewcommand{\literalblockname}{Code source}}

\addto\captionsfrench{\renewcommand{\literalblockcontinuedname}{continued from previous page}}
\addto\captionsfrench{\renewcommand{\literalblockcontinuesname}{continues on next page}}

\addto\extrasfrench{\def\pageautorefname{page}}

\setcounter{tocdepth}{1}



\title{TetrisBot Documentation}
\date{mai 31, 2019}
\release{1.0}
\author{F.Muller - L.Ponton}
\newcommand{\sphinxlogo}{\vbox{}}
\renewcommand{\releasename}{Version}
\makeindex

\begin{document}

\maketitle
\sphinxtableofcontents
\phantomsection\label{\detokenize{index::doc}}



\chapter{Le moteur de jeu}
\label{\detokenize{index:welcome-to-tetrisbot-s-documentation}}\label{\detokenize{index:le-moteur-de-jeu}}

\section{Module tetramino.py}
\label{\detokenize{index:module-tetramino}}\label{\detokenize{index:module-tetramino-py}}\index{tetramino (module)}\index{Tetramino (classe dans tetramino)}

\begin{fulllineitems}
\phantomsection\label{\detokenize{index:tetramino.Tetramino}}\pysiglinewithargsret{\sphinxbfcode{class }\sphinxcode{tetramino.}\sphinxbfcode{Tetramino}}{\emph{id}, \emph{rotations}, \emph{corners}}{}
Classe de gestion des pièces
\index{copy() (méthode tetramino.Tetramino)}

\begin{fulllineitems}
\phantomsection\label{\detokenize{index:tetramino.Tetramino.copy}}\pysiglinewithargsret{\sphinxbfcode{copy}}{}{}
Renvoie une copie de la pièce

\end{fulllineitems}

\index{getBottomCells() (méthode tetramino.Tetramino)}

\begin{fulllineitems}
\phantomsection\label{\detokenize{index:tetramino.Tetramino.getBottomCells}}\pysiglinewithargsret{\sphinxbfcode{getBottomCells}}{}{}
Renvoie les coordonnées des cellules les plus en bas

\end{fulllineitems}

\index{getCorners() (méthode tetramino.Tetramino)}

\begin{fulllineitems}
\phantomsection\label{\detokenize{index:tetramino.Tetramino.getCorners}}\pysiglinewithargsret{\sphinxbfcode{getCorners}}{}{}
Renvoie les coordonées des coins de la pièce

\end{fulllineitems}

\index{getLowerCell() (méthode tetramino.Tetramino)}

\begin{fulllineitems}
\phantomsection\label{\detokenize{index:tetramino.Tetramino.getLowerCell}}\pysiglinewithargsret{\sphinxbfcode{getLowerCell}}{\emph{j}}{}
Renvoie la ligne de la cellule la plus en bas dans la colonne j

\end{fulllineitems}

\index{rotate() (méthode tetramino.Tetramino)}

\begin{fulllineitems}
\phantomsection\label{\detokenize{index:tetramino.Tetramino.rotate}}\pysiglinewithargsret{\sphinxbfcode{rotate}}{\emph{direction='H'}}{}
Tourne la pièce dans la direction donnée 
- “H” : Sens Horaire
- “T” : Sens Trigo

\end{fulllineitems}

\index{setRotation() (méthode tetramino.Tetramino)}

\begin{fulllineitems}
\phantomsection\label{\detokenize{index:tetramino.Tetramino.setRotation}}\pysiglinewithargsret{\sphinxbfcode{setRotation}}{\emph{i}}{}
Tourne directement une pièce

\end{fulllineitems}

\index{toArray() (méthode tetramino.Tetramino)}

\begin{fulllineitems}
\phantomsection\label{\detokenize{index:tetramino.Tetramino.toArray}}\pysiglinewithargsret{\sphinxbfcode{toArray}}{}{}
Renvoie la représentation du bloc sous forme de matrice carrée

\end{fulllineitems}


\end{fulllineitems}



\section{Module board.py}
\label{\detokenize{index:module-board}}\label{\detokenize{index:module-board-py}}\index{board (module)}\index{Board (classe dans board)}

\begin{fulllineitems}
\phantomsection\label{\detokenize{index:board.Board}}\pysiglinewithargsret{\sphinxbfcode{class }\sphinxcode{board.}\sphinxbfcode{Board}}{\emph{width=10}, \emph{height=22}}{}
Classe de gestion de la grille
\index{columnHeight() (méthode board.Board)}

\begin{fulllineitems}
\phantomsection\label{\detokenize{index:board.Board.columnHeight}}\pysiglinewithargsret{\sphinxbfcode{columnHeight}}{\emph{j}}{}
Renvoie la hauteur de la colonne j 
Attention, cette fonction renvoie la hauteur et non l’indice de la dernière pièce

\end{fulllineitems}

\index{copy() (méthode board.Board)}

\begin{fulllineitems}
\phantomsection\label{\detokenize{index:board.Board.copy}}\pysiglinewithargsret{\sphinxbfcode{copy}}{}{}
Renvoie une copie de la grille

\end{fulllineitems}

\index{decodeFromInt() (méthode board.Board)}

\begin{fulllineitems}
\phantomsection\label{\detokenize{index:board.Board.decodeFromInt}}\pysiglinewithargsret{\sphinxbfcode{decodeFromInt}}{\emph{n}}{}
Renvoie un tableau à partir d’un code entier

\end{fulllineitems}

\index{emptyCell() (méthode board.Board)}

\begin{fulllineitems}
\phantomsection\label{\detokenize{index:board.Board.emptyCell}}\pysiglinewithargsret{\sphinxbfcode{emptyCell}}{\emph{i}, \emph{j}}{}
Vide la cellule (i,j)

\end{fulllineitems}

\index{encodeToInt() (méthode board.Board)}

\begin{fulllineitems}
\phantomsection\label{\detokenize{index:board.Board.encodeToInt}}\pysiglinewithargsret{\sphinxbfcode{encodeToInt}}{}{}
Renvoie une représentation de la grille sous la forme d’un entier

\end{fulllineitems}

\index{getBumpiness() (méthode board.Board)}

\begin{fulllineitems}
\phantomsection\label{\detokenize{index:board.Board.getBumpiness}}\pysiglinewithargsret{\sphinxbfcode{getBumpiness}}{}{}
Renvoie la somme des valeurs absolues des différences
de hauteurs entre les colonnes consécutives

\end{fulllineitems}

\index{getCell() (méthode board.Board)}

\begin{fulllineitems}
\phantomsection\label{\detokenize{index:board.Board.getCell}}\pysiglinewithargsret{\sphinxbfcode{getCell}}{\emph{i}, \emph{j}}{}
Renvoie le contenu de la cellule (i,j)

\end{fulllineitems}

\index{getColumnHeights() (méthode board.Board)}

\begin{fulllineitems}
\phantomsection\label{\detokenize{index:board.Board.getColumnHeights}}\pysiglinewithargsret{\sphinxbfcode{getColumnHeights}}{}{}
Renvoie la liste des hauteurs des colonnes

\end{fulllineitems}

\index{getMaxHeight() (méthode board.Board)}

\begin{fulllineitems}
\phantomsection\label{\detokenize{index:board.Board.getMaxHeight}}\pysiglinewithargsret{\sphinxbfcode{getMaxHeight}}{}{}
Renvoie la hauteur maximum des pièces du jeu

\end{fulllineitems}

\index{getNbHoles() (méthode board.Board)}

\begin{fulllineitems}
\phantomsection\label{\detokenize{index:board.Board.getNbHoles}}\pysiglinewithargsret{\sphinxbfcode{getNbHoles}}{}{}
Renvoie le nombre de trous dans la grille
(en fait ici juste les cases dominées)

\end{fulllineitems}

\index{getSumHeights() (méthode board.Board)}

\begin{fulllineitems}
\phantomsection\label{\detokenize{index:board.Board.getSumHeights}}\pysiglinewithargsret{\sphinxbfcode{getSumHeights}}{}{}
Renvoie la somme des hauteurs des colonnes

\end{fulllineitems}

\index{isCellEmpty() (méthode board.Board)}

\begin{fulllineitems}
\phantomsection\label{\detokenize{index:board.Board.isCellEmpty}}\pysiglinewithargsret{\sphinxbfcode{isCellEmpty}}{\emph{i}, \emph{j}}{}
Teste si la cellule (i,j) est vide

\end{fulllineitems}

\index{isDominated() (méthode board.Board)}

\begin{fulllineitems}
\phantomsection\label{\detokenize{index:board.Board.isDominated}}\pysiglinewithargsret{\sphinxbfcode{isDominated}}{\emph{i}, \emph{j}}{}
Teste si une case est vide et est dominée par une case au-dessus

\end{fulllineitems}

\index{isLineFull() (méthode board.Board)}

\begin{fulllineitems}
\phantomsection\label{\detokenize{index:board.Board.isLineFull}}\pysiglinewithargsret{\sphinxbfcode{isLineFull}}{\emph{i}}{}
Teste si une ligne est pleine

\end{fulllineitems}

\index{npBinaryRepresentation() (méthode board.Board)}

\begin{fulllineitems}
\phantomsection\label{\detokenize{index:board.Board.npBinaryRepresentation}}\pysiglinewithargsret{\sphinxbfcode{npBinaryRepresentation}}{}{}
Renvoie un numpy array contenant la grille

\end{fulllineitems}

\index{printInfos() (méthode board.Board)}

\begin{fulllineitems}
\phantomsection\label{\detokenize{index:board.Board.printInfos}}\pysiglinewithargsret{\sphinxbfcode{printInfos}}{}{}
Affiche les infos de la grille (pour tests)

\end{fulllineitems}

\index{processLines() (méthode board.Board)}

\begin{fulllineitems}
\phantomsection\label{\detokenize{index:board.Board.processLines}}\pysiglinewithargsret{\sphinxbfcode{processLines}}{}{}
Enlève les lignes finies et renvoie le nombre de lignes enlevées

\end{fulllineitems}

\index{removeLine() (méthode board.Board)}

\begin{fulllineitems}
\phantomsection\label{\detokenize{index:board.Board.removeLine}}\pysiglinewithargsret{\sphinxbfcode{removeLine}}{\emph{i}}{}
Supprime la ligne i

\end{fulllineitems}

\index{setCell() (méthode board.Board)}

\begin{fulllineitems}
\phantomsection\label{\detokenize{index:board.Board.setCell}}\pysiglinewithargsret{\sphinxbfcode{setCell}}{\emph{i}, \emph{j}, \emph{value}}{}
Met value dans la cellule (i,j)

\end{fulllineitems}

\index{updateStats() (méthode board.Board)}

\begin{fulllineitems}
\phantomsection\label{\detokenize{index:board.Board.updateStats}}\pysiglinewithargsret{\sphinxbfcode{updateStats}}{}{}
Met à jour tous les paramètres de la grille

\end{fulllineitems}


\end{fulllineitems}



\section{Module tetris\_engine.py}
\label{\detokenize{index:module-tetris-engine-py}}\label{\detokenize{index:module-tetris_engine}}\index{tetris\_engine (module)}\index{TetrisEngine (classe dans tetris\_engine)}

\begin{fulllineitems}
\phantomsection\label{\detokenize{index:tetris_engine.TetrisEngine}}\pysiglinewithargsret{\sphinxbfcode{class }\sphinxcode{tetris\_engine.}\sphinxbfcode{TetrisEngine}}{\emph{getMove=\textless{}function \textless{}lambda\textgreater{}\textgreater{}, width=10, height=22, max\_blocks=0, base\_blocks\_bag={[}\textless{}tetramino.Tetramino instance\textgreater{}, \textless{}tetramino.Tetramino instance\textgreater{}, \textless{}tetramino.Tetramino instance\textgreater{}, \textless{}tetramino.Tetramino instance\textgreater{}, \textless{}tetramino.Tetramino instance\textgreater{}, \textless{}tetramino.Tetramino instance\textgreater{}, \textless{}tetramino.Tetramino instance\textgreater{}{]}, temporisation=0, silent=False, random\_generator\_seed=None, agent\_name='', agent\_description=''}}{}
Classe gérant le moteur de jeu
\index{canPlaceBlockDirect() (méthode tetris\_engine.TetrisEngine)}

\begin{fulllineitems}
\phantomsection\label{\detokenize{index:tetris_engine.TetrisEngine.canPlaceBlockDirect}}\pysiglinewithargsret{\sphinxbfcode{canPlaceBlockDirect}}{\emph{column}, \emph{rotation}}{}
Teste si on peut placer directement un bloc 
dans la colonne et la rotation donnée

\end{fulllineitems}

\index{copy() (méthode tetris\_engine.TetrisEngine)}

\begin{fulllineitems}
\phantomsection\label{\detokenize{index:tetris_engine.TetrisEngine.copy}}\pysiglinewithargsret{\sphinxbfcode{copy}}{}{}
Renvoie une copie de l’environnement

\end{fulllineitems}

\index{dropBlock() (méthode tetris\_engine.TetrisEngine)}

\begin{fulllineitems}
\phantomsection\label{\detokenize{index:tetris_engine.TetrisEngine.dropBlock}}\pysiglinewithargsret{\sphinxbfcode{dropBlock}}{}{}
Fait tomber le bloc en bas

\end{fulllineitems}

\index{eraseBlock() (méthode tetris\_engine.TetrisEngine)}

\begin{fulllineitems}
\phantomsection\label{\detokenize{index:tetris_engine.TetrisEngine.eraseBlock}}\pysiglinewithargsret{\sphinxbfcode{eraseBlock}}{}{}
Efface le bloc de sa position

\end{fulllineitems}

\index{generateNewBlock() (méthode tetris\_engine.TetrisEngine)}

\begin{fulllineitems}
\phantomsection\label{\detokenize{index:tetris_engine.TetrisEngine.generateNewBlock}}\pysiglinewithargsret{\sphinxbfcode{generateNewBlock}}{}{}
Remplace le bloc courant par le suivant et fabrique un nouveau bloc suivant

\end{fulllineitems}

\index{generateNewBlockBag() (méthode tetris\_engine.TetrisEngine)}

\begin{fulllineitems}
\phantomsection\label{\detokenize{index:tetris_engine.TetrisEngine.generateNewBlockBag}}\pysiglinewithargsret{\sphinxbfcode{generateNewBlockBag}}{}{}
Génère un nouveau sac de pièces

\end{fulllineitems}

\index{getNewBlock() (méthode tetris\_engine.TetrisEngine)}

\begin{fulllineitems}
\phantomsection\label{\detokenize{index:tetris_engine.TetrisEngine.getNewBlock}}\pysiglinewithargsret{\sphinxbfcode{getNewBlock}}{}{}
Met un nouveau bloc en jeu

\end{fulllineitems}

\index{getPossibleMovesDirect() (méthode tetris\_engine.TetrisEngine)}

\begin{fulllineitems}
\phantomsection\label{\detokenize{index:tetris_engine.TetrisEngine.getPossibleMovesDirect}}\pysiglinewithargsret{\sphinxbfcode{getPossibleMovesDirect}}{}{}
Renvoie la liste de tous les placements directs possibles
sous la forme de tuples (column, rotation)

\end{fulllineitems}

\index{getScoreFromLines() (méthode tetris\_engine.TetrisEngine)}

\begin{fulllineitems}
\phantomsection\label{\detokenize{index:tetris_engine.TetrisEngine.getScoreFromLines}}\pysiglinewithargsret{\sphinxbfcode{getScoreFromLines}}{\emph{nb\_lines}}{}
Renvoie le score suivant le nombre de lignes faites

\end{fulllineitems}

\index{getStrAgentName() (méthode tetris\_engine.TetrisEngine)}

\begin{fulllineitems}
\phantomsection\label{\detokenize{index:tetris_engine.TetrisEngine.getStrAgentName}}\pysiglinewithargsret{\sphinxbfcode{getStrAgentName}}{}{}
Renvoie la chaîne contenant les nom et la description de l’agent

\end{fulllineitems}

\index{getStrInfos() (méthode tetris\_engine.TetrisEngine)}

\begin{fulllineitems}
\phantomsection\label{\detokenize{index:tetris_engine.TetrisEngine.getStrInfos}}\pysiglinewithargsret{\sphinxbfcode{getStrInfos}}{}{}
Renvoie une chaîne contenant les informations de la partie

\end{fulllineitems}

\index{getStrNextBlock() (méthode tetris\_engine.TetrisEngine)}

\begin{fulllineitems}
\phantomsection\label{\detokenize{index:tetris_engine.TetrisEngine.getStrNextBlock}}\pysiglinewithargsret{\sphinxbfcode{getStrNextBlock}}{}{}
Renvoie une chaîne pour afficher le bloc suivant

\end{fulllineitems}

\index{isEndGame() (méthode tetris\_engine.TetrisEngine)}

\begin{fulllineitems}
\phantomsection\label{\detokenize{index:tetris_engine.TetrisEngine.isEndGame}}\pysiglinewithargsret{\sphinxbfcode{isEndGame}}{}{}
Teste la fin du jeu

\end{fulllineitems}

\index{isMoveInGrid() (méthode tetris\_engine.TetrisEngine)}

\begin{fulllineitems}
\phantomsection\label{\detokenize{index:tetris_engine.TetrisEngine.isMoveInGrid}}\pysiglinewithargsret{\sphinxbfcode{isMoveInGrid}}{\emph{block}, \emph{new\_position}}{}
Teste si le bloc reste dans la grille

\end{fulllineitems}

\index{isMoveOnFreeCells() (méthode tetris\_engine.TetrisEngine)}

\begin{fulllineitems}
\phantomsection\label{\detokenize{index:tetris_engine.TetrisEngine.isMoveOnFreeCells}}\pysiglinewithargsret{\sphinxbfcode{isMoveOnFreeCells}}{\emph{block}, \emph{new\_position}}{}
Teste si les nouvelles cases occupées par le bloc sont vides

\end{fulllineitems}

\index{isMoveValid() (méthode tetris\_engine.TetrisEngine)}

\begin{fulllineitems}
\phantomsection\label{\detokenize{index:tetris_engine.TetrisEngine.isMoveValid}}\pysiglinewithargsret{\sphinxbfcode{isMoveValid}}{\emph{block}, \emph{new\_position}}{}
Teste si une position est valide pour un bloc

\end{fulllineitems}

\index{minimalCopy() (méthode tetris\_engine.TetrisEngine)}

\begin{fulllineitems}
\phantomsection\label{\detokenize{index:tetris_engine.TetrisEngine.minimalCopy}}\pysiglinewithargsret{\sphinxbfcode{minimalCopy}}{}{}
Renvoie une copie minimale de l’environnement
(grilles et pièces) pour tester les différents coups

\end{fulllineitems}

\index{moveBlock() (méthode tetris\_engine.TetrisEngine)}

\begin{fulllineitems}
\phantomsection\label{\detokenize{index:tetris_engine.TetrisEngine.moveBlock}}\pysiglinewithargsret{\sphinxbfcode{moveBlock}}{\emph{block}, \emph{new\_position}}{}
Déplace un bloc vers une nouvelle position

\end{fulllineitems}

\index{moveBlockInDirection() (méthode tetris\_engine.TetrisEngine)}

\begin{fulllineitems}
\phantomsection\label{\detokenize{index:tetris_engine.TetrisEngine.moveBlockInDirection}}\pysiglinewithargsret{\sphinxbfcode{moveBlockInDirection}}{\emph{direction=''}}{}
Déplace le bloc dans une direction :
- “L” : Left
- “R” : Right
- “” : Vers le bas par défaut

\end{fulllineitems}

\index{placeBlock() (méthode tetris\_engine.TetrisEngine)}

\begin{fulllineitems}
\phantomsection\label{\detokenize{index:tetris_engine.TetrisEngine.placeBlock}}\pysiglinewithargsret{\sphinxbfcode{placeBlock}}{\emph{block}, \emph{position}}{}
Place un bloc dans une position

\end{fulllineitems}

\index{placeBlockDirect() (méthode tetris\_engine.TetrisEngine)}

\begin{fulllineitems}
\phantomsection\label{\detokenize{index:tetris_engine.TetrisEngine.placeBlockDirect}}\pysiglinewithargsret{\sphinxbfcode{placeBlockDirect}}{\emph{column}, \emph{rotation}}{}
Place directement un bloc dans la colonne et la rotation donnée

\end{fulllineitems}

\index{playCommand() (méthode tetris\_engine.TetrisEngine)}

\begin{fulllineitems}
\phantomsection\label{\detokenize{index:tetris_engine.TetrisEngine.playCommand}}\pysiglinewithargsret{\sphinxbfcode{playCommand}}{\emph{command='N'}}{}
Joue une commande :
- “L” : Move Left
- “R” : Move Right
- “D” : Drop
- “N” : Nothing (descend d’une case)
- “H” : Rotate Hours direction
- “T” : Rotate Trigo direction
- “P:j:r” : Place block in column j with rotation r
- “S” : Restart
- “Q” : Quit

\end{fulllineitems}

\index{printRightColumn() (méthode tetris\_engine.TetrisEngine)}

\begin{fulllineitems}
\phantomsection\label{\detokenize{index:tetris_engine.TetrisEngine.printRightColumn}}\pysiglinewithargsret{\sphinxbfcode{printRightColumn}}{}{}
Renvoie la chaîne correspondant au côté droit de l’affichage

\end{fulllineitems}

\index{rotateBlockInDirection() (méthode tetris\_engine.TetrisEngine)}

\begin{fulllineitems}
\phantomsection\label{\detokenize{index:tetris_engine.TetrisEngine.rotateBlockInDirection}}\pysiglinewithargsret{\sphinxbfcode{rotateBlockInDirection}}{\emph{direction='H'}}{}
Tourne le bloc dans une direction :
- “H” : Sens Horaire
- “T” : Sens Trigo

\end{fulllineitems}

\index{run() (méthode tetris\_engine.TetrisEngine)}

\begin{fulllineitems}
\phantomsection\label{\detokenize{index:tetris_engine.TetrisEngine.run}}\pysiglinewithargsret{\sphinxbfcode{run}}{}{}
Boucle principale du jeu

\end{fulllineitems}

\index{setBlockInitPosition() (méthode tetris\_engine.TetrisEngine)}

\begin{fulllineitems}
\phantomsection\label{\detokenize{index:tetris_engine.TetrisEngine.setBlockInitPosition}}\pysiglinewithargsret{\sphinxbfcode{setBlockInitPosition}}{}{}
Position initiale pour un nouveau bloc

\end{fulllineitems}

\index{updateTimes() (méthode tetris\_engine.TetrisEngine)}

\begin{fulllineitems}
\phantomsection\label{\detokenize{index:tetris_engine.TetrisEngine.updateTimes}}\pysiglinewithargsret{\sphinxbfcode{updateTimes}}{\emph{start\_time}}{}
Met à jour les chronos

\end{fulllineitems}


\end{fulllineitems}

\index{getrandbits() (dans le module tetris\_engine)}

\begin{fulllineitems}
\phantomsection\label{\detokenize{index:tetris_engine.getrandbits}}\pysiglinewithargsret{\sphinxcode{tetris\_engine.}\sphinxbfcode{getrandbits}}{\emph{k}}{{ $\rightarrow$ x.  Generates a long int with k random bits.}}
\end{fulllineitems}

\index{random() (dans le module tetris\_engine)}

\begin{fulllineitems}
\phantomsection\label{\detokenize{index:tetris_engine.random}}\pysiglinewithargsret{\sphinxcode{tetris\_engine.}\sphinxbfcode{random}}{}{{ $\rightarrow$ x in the interval {[}0, 1).}}
\end{fulllineitems}



\chapter{Les agents}
\label{\detokenize{index:les-agents}}

\section{Module agent.py}
\label{\detokenize{index:module-agent}}\label{\detokenize{index:module-agent-py}}\index{agent (module)}\index{Agent (classe dans agent)}

\begin{fulllineitems}
\phantomsection\label{\detokenize{index:agent.Agent}}\pysiglinewithargsret{\sphinxbfcode{class }\sphinxcode{agent.}\sphinxbfcode{Agent}}{\emph{name=''}, \emph{description=''}}{}
La classe de base des agents
\index{allMovesStats() (méthode agent.Agent)}

\begin{fulllineitems}
\phantomsection\label{\detokenize{index:agent.Agent.allMovesStats}}\pysiglinewithargsret{\sphinxbfcode{allMovesStats}}{}{}
Renvoie un dictionnaire contenant les stats de chaque mouvement possible
Les clefs sont les mouvements et les valeurs sont les stats de ce mouvement

\end{fulllineitems}

\index{commandFromMove() (méthode agent.Agent)}

\begin{fulllineitems}
\phantomsection\label{\detokenize{index:agent.Agent.commandFromMove}}\pysiglinewithargsret{\sphinxbfcode{commandFromMove}}{\emph{move}}{}
Renvoie la commande d’un mouvement à passer à l’engine

\end{fulllineitems}

\index{getMoveStats() (méthode agent.Agent)}

\begin{fulllineitems}
\phantomsection\label{\detokenize{index:agent.Agent.getMoveStats}}\pysiglinewithargsret{\sphinxbfcode{getMoveStats}}{\emph{move}}{}
Remplit le dictionnaire contenant les statistiques de la grille
après que le mouvement move ait été joué

\end{fulllineitems}


\end{fulllineitems}

\index{benchPlayer() (dans le module agent)}

\begin{fulllineitems}
\phantomsection\label{\detokenize{index:agent.benchPlayer}}\pysiglinewithargsret{\sphinxcode{agent.}\sphinxbfcode{benchPlayer}}{\emph{player\_init}, \emph{nb\_samples=100}, \emph{max\_blocks=0}}{}
Réalise un bench de AgentToTest en jouant nb\_samples parties

\end{fulllineitems}

\index{getrandbits() (dans le module agent)}

\begin{fulllineitems}
\phantomsection\label{\detokenize{index:agent.getrandbits}}\pysiglinewithargsret{\sphinxcode{agent.}\sphinxbfcode{getrandbits}}{\emph{k}}{{ $\rightarrow$ x.  Generates a long int with k random bits.}}
\end{fulllineitems}

\index{playGame() (dans le module agent)}

\begin{fulllineitems}
\phantomsection\label{\detokenize{index:agent.playGame}}\pysiglinewithargsret{\sphinxcode{agent.}\sphinxbfcode{playGame}}{\emph{player\_init}, \emph{temporisation=0.1}}{}
Lance des parties avec l’agent

\end{fulllineitems}

\index{plotBenchPlayer() (dans le module agent)}

\begin{fulllineitems}
\phantomsection\label{\detokenize{index:agent.plotBenchPlayer}}\pysiglinewithargsret{\sphinxcode{agent.}\sphinxbfcode{plotBenchPlayer}}{\emph{player\_init}, \emph{nb\_samples}, \emph{nb\_bars=10}, \emph{max\_blocks=0}}{}
Réalise un bench de AgentToTest en jouant nb\_samples parties
Affiche les résultats sous la forme d’un histogramme avec nb\_bars classes

\end{fulllineitems}

\index{random() (dans le module agent)}

\begin{fulllineitems}
\phantomsection\label{\detokenize{index:agent.random}}\pysiglinewithargsret{\sphinxcode{agent.}\sphinxbfcode{random}}{}{{ $\rightarrow$ x in the interval {[}0, 1).}}
\end{fulllineitems}



\section{Module agent\_human.py}
\label{\detokenize{index:module-agent_human}}\label{\detokenize{index:module-agent-human-py}}\index{agent\_human (module)}\index{AgentHuman (classe dans agent\_human)}

\begin{fulllineitems}
\phantomsection\label{\detokenize{index:agent_human.AgentHuman}}\pysiglinewithargsret{\sphinxbfcode{class }\sphinxcode{agent\_human.}\sphinxbfcode{AgentHuman}}{\emph{temporisation=0}, \emph{silent=False}}{}
Agent humain
\index{getMove() (méthode agent\_human.AgentHuman)}

\begin{fulllineitems}
\phantomsection\label{\detokenize{index:agent_human.AgentHuman.getMove}}\pysiglinewithargsret{\sphinxbfcode{getMove}}{}{}
Entre un mouvement à jouer

\end{fulllineitems}


\end{fulllineitems}

\index{getrandbits() (dans le module agent\_human)}

\begin{fulllineitems}
\phantomsection\label{\detokenize{index:agent_human.getrandbits}}\pysiglinewithargsret{\sphinxcode{agent\_human.}\sphinxbfcode{getrandbits}}{\emph{k}}{{ $\rightarrow$ x.  Generates a long int with k random bits.}}
\end{fulllineitems}

\index{random() (dans le module agent\_human)}

\begin{fulllineitems}
\phantomsection\label{\detokenize{index:agent_human.random}}\pysiglinewithargsret{\sphinxcode{agent\_human.}\sphinxbfcode{random}}{}{{ $\rightarrow$ x in the interval {[}0, 1).}}
\end{fulllineitems}



\section{Module agent\_random1.py}
\label{\detokenize{index:module-agent_random1}}\label{\detokenize{index:module-agent-random1-py}}\index{agent\_random1 (module)}\index{AgentRandom1 (classe dans agent\_random1)}

\begin{fulllineitems}
\phantomsection\label{\detokenize{index:agent_random1.AgentRandom1}}\pysiglinewithargsret{\sphinxbfcode{class }\sphinxcode{agent\_random1.}\sphinxbfcode{AgentRandom1}}{\emph{temporisation=0.1}, \emph{silent=False}}{}
Agent aléatoire jouant avec les touches du clavier
\index{getMove() (méthode agent\_random1.AgentRandom1)}

\begin{fulllineitems}
\phantomsection\label{\detokenize{index:agent_random1.AgentRandom1.getMove}}\pysiglinewithargsret{\sphinxbfcode{getMove}}{}{}
Renvoie un mouvement de touche aléatoire

\end{fulllineitems}


\end{fulllineitems}

\index{getrandbits() (dans le module agent\_random1)}

\begin{fulllineitems}
\phantomsection\label{\detokenize{index:agent_random1.getrandbits}}\pysiglinewithargsret{\sphinxcode{agent\_random1.}\sphinxbfcode{getrandbits}}{\emph{k}}{{ $\rightarrow$ x.  Generates a long int with k random bits.}}
\end{fulllineitems}

\index{random() (dans le module agent\_random1)}

\begin{fulllineitems}
\phantomsection\label{\detokenize{index:agent_random1.random}}\pysiglinewithargsret{\sphinxcode{agent\_random1.}\sphinxbfcode{random}}{}{{ $\rightarrow$ x in the interval {[}0, 1).}}
\end{fulllineitems}



\section{Module agent\_random2.py}
\label{\detokenize{index:module-agent_random2}}\label{\detokenize{index:module-agent-random2-py}}\index{agent\_random2 (module)}\index{AgentRandom2 (classe dans agent\_random2)}

\begin{fulllineitems}
\phantomsection\label{\detokenize{index:agent_random2.AgentRandom2}}\pysiglinewithargsret{\sphinxbfcode{class }\sphinxcode{agent\_random2.}\sphinxbfcode{AgentRandom2}}{\emph{temporisation=0.1}, \emph{silent=False}}{}
Agent aléatoire jouant directement les pièces
\index{getMove() (méthode agent\_random2.AgentRandom2)}

\begin{fulllineitems}
\phantomsection\label{\detokenize{index:agent_random2.AgentRandom2.getMove}}\pysiglinewithargsret{\sphinxbfcode{getMove}}{}{}
Renvoie un mouvement direct aléatoire

\end{fulllineitems}


\end{fulllineitems}

\index{getrandbits() (dans le module agent\_random2)}

\begin{fulllineitems}
\phantomsection\label{\detokenize{index:agent_random2.getrandbits}}\pysiglinewithargsret{\sphinxcode{agent\_random2.}\sphinxbfcode{getrandbits}}{\emph{k}}{{ $\rightarrow$ x.  Generates a long int with k random bits.}}
\end{fulllineitems}

\index{random() (dans le module agent\_random2)}

\begin{fulllineitems}
\phantomsection\label{\detokenize{index:agent_random2.random}}\pysiglinewithargsret{\sphinxcode{agent\_random2.}\sphinxbfcode{random}}{}{{ $\rightarrow$ x in the interval {[}0, 1).}}
\end{fulllineitems}



\section{Module agent\_filtering.py}
\label{\detokenize{index:module-agent_filtering}}\label{\detokenize{index:module-agent-filtering-py}}\index{agent\_filtering (module)}\index{AgentFiltering (classe dans agent\_filtering)}

\begin{fulllineitems}
\phantomsection\label{\detokenize{index:agent_filtering.AgentFiltering}}\pysiglinewithargsret{\sphinxbfcode{class }\sphinxcode{agent\_filtering.}\sphinxbfcode{AgentFiltering}}{\emph{temporisation=0.1, silent=False, order={[}'holes', 'sum\_heights', 'bumpiness', 'lines'{]}}}{}
Agent procédant par filtrage des coups
\index{filterMoves() (méthode agent\_filtering.AgentFiltering)}

\begin{fulllineitems}
\phantomsection\label{\detokenize{index:agent_filtering.AgentFiltering.filterMoves}}\pysiglinewithargsret{\sphinxbfcode{filterMoves}}{\emph{stat}, \emph{value}}{}
Filtre les mouvements en récupérant uniquement ceux
dont la stat vaut value

\end{fulllineitems}

\index{getMove() (méthode agent\_filtering.AgentFiltering)}

\begin{fulllineitems}
\phantomsection\label{\detokenize{index:agent_filtering.AgentFiltering.getMove}}\pysiglinewithargsret{\sphinxbfcode{getMove}}{}{}
Optimisation en filtrant successivement les mouvements 
suivant les différentes stats

\end{fulllineitems}

\index{maxStat() (méthode agent\_filtering.AgentFiltering)}

\begin{fulllineitems}
\phantomsection\label{\detokenize{index:agent_filtering.AgentFiltering.maxStat}}\pysiglinewithargsret{\sphinxbfcode{maxStat}}{\emph{stat}}{}
Renvoie la valeur maxi d’une stat

\end{fulllineitems}

\index{minStat() (méthode agent\_filtering.AgentFiltering)}

\begin{fulllineitems}
\phantomsection\label{\detokenize{index:agent_filtering.AgentFiltering.minStat}}\pysiglinewithargsret{\sphinxbfcode{minStat}}{\emph{stat}}{}
Renvoie la valeur mini d’une stat

\end{fulllineitems}


\end{fulllineitems}

\index{getrandbits() (dans le module agent\_filtering)}

\begin{fulllineitems}
\phantomsection\label{\detokenize{index:agent_filtering.getrandbits}}\pysiglinewithargsret{\sphinxcode{agent\_filtering.}\sphinxbfcode{getrandbits}}{\emph{k}}{{ $\rightarrow$ x.  Generates a long int with k random bits.}}
\end{fulllineitems}

\index{random() (dans le module agent\_filtering)}

\begin{fulllineitems}
\phantomsection\label{\detokenize{index:agent_filtering.random}}\pysiglinewithargsret{\sphinxcode{agent\_filtering.}\sphinxbfcode{random}}{}{{ $\rightarrow$ x in the interval {[}0, 1).}}
\end{fulllineitems}



\section{Module agent\_evaluation.py}
\label{\detokenize{index:module-agent_evaluation}}\label{\detokenize{index:module-agent-evaluation-py}}\index{agent\_evaluation (module)}\index{AgentEvaluation (classe dans agent\_evaluation)}

\begin{fulllineitems}
\phantomsection\label{\detokenize{index:agent_evaluation.AgentEvaluation}}\pysiglinewithargsret{\sphinxbfcode{class }\sphinxcode{agent\_evaluation.}\sphinxbfcode{AgentEvaluation}}{\emph{eval\_coeffs={[}0.548, 0.5218, 0.6267, 0.1862{]}, temporisation=0.1, silent=False}}{}
Agent procédant par évaluation des coups
\index{getMove() (méthode agent\_evaluation.AgentEvaluation)}

\begin{fulllineitems}
\phantomsection\label{\detokenize{index:agent_evaluation.AgentEvaluation.getMove}}\pysiglinewithargsret{\sphinxbfcode{getMove}}{}{}
Optimisation à partir de la fonction d’évaluation

\end{fulllineitems}

\index{moveEvaluation() (méthode agent\_evaluation.AgentEvaluation)}

\begin{fulllineitems}
\phantomsection\label{\detokenize{index:agent_evaluation.AgentEvaluation.moveEvaluation}}\pysiglinewithargsret{\sphinxbfcode{moveEvaluation}}{\emph{move}}{}
Évalue le mouvement move=(j, r)

\end{fulllineitems}


\end{fulllineitems}

\index{getrandbits() (dans le module agent\_evaluation)}

\begin{fulllineitems}
\phantomsection\label{\detokenize{index:agent_evaluation.getrandbits}}\pysiglinewithargsret{\sphinxcode{agent\_evaluation.}\sphinxbfcode{getrandbits}}{\emph{k}}{{ $\rightarrow$ x.  Generates a long int with k random bits.}}
\end{fulllineitems}

\index{playGameWithAgentEvaluation() (dans le module agent\_evaluation)}

\begin{fulllineitems}
\phantomsection\label{\detokenize{index:agent_evaluation.playGameWithAgentEvaluation}}\pysiglinewithargsret{\sphinxcode{agent\_evaluation.}\sphinxbfcode{playGameWithAgentEvaluation}}{\emph{coeffs}, \emph{temporisation=0}}{}
Joue des parties avec l’agent par évaluation et les coeffs donnés

\end{fulllineitems}

\index{random() (dans le module agent\_evaluation)}

\begin{fulllineitems}
\phantomsection\label{\detokenize{index:agent_evaluation.random}}\pysiglinewithargsret{\sphinxcode{agent\_evaluation.}\sphinxbfcode{random}}{}{{ $\rightarrow$ x in the interval {[}0, 1).}}
\end{fulllineitems}



\chapter{Algorithmes génétiques}
\label{\detokenize{index:algorithmes-genetiques}}

\section{Module ag\_optimizer.py}
\label{\detokenize{index:module-ag_optimizer}}\label{\detokenize{index:module-ag-optimizer-py}}\index{ag\_optimizer (module)}\index{AGOptimizer (classe dans ag\_optimizer)}

\begin{fulllineitems}
\phantomsection\label{\detokenize{index:ag_optimizer.AGOptimizer}}\pysiglinewithargsret{\sphinxbfcode{class }\sphinxcode{ag\_optimizer.}\sphinxbfcode{AGOptimizer}}{\emph{population\_size=20}, \emph{nb\_generations=2}, \emph{nb\_bits=16}, \emph{max\_nb\_blocks=5}, \emph{nb\_games\_played=1}, \emph{proba\_mutation=0.05}, \emph{mutation\_rate=0.2}, \emph{percentage\_for\_tournament=0.1}, \emph{percentage\_new\_offspring=0.3}, \emph{elitism\_percentage=0.2}, \emph{vector\_encoding='float'}, \emph{parents\_selection\_method='tournament'}, \emph{old\_generation\_policy='best'}, \emph{evaluation\_criteria='lines'}}{}
Optimisation des coefficients par algorithme génétique
\index{crossover() (méthode ag\_optimizer.AGOptimizer)}

\begin{fulllineitems}
\phantomsection\label{\detokenize{index:ag_optimizer.AGOptimizer.crossover}}\pysiglinewithargsret{\sphinxbfcode{crossover}}{\emph{parent1}, \emph{parent2}}{}
Renvoie le ou les enfants de parent1 et parent2

\end{fulllineitems}

\index{deleteWorst() (méthode ag\_optimizer.AGOptimizer)}

\begin{fulllineitems}
\phantomsection\label{\detokenize{index:ag_optimizer.AGOptimizer.deleteWorst}}\pysiglinewithargsret{\sphinxbfcode{deleteWorst}}{}{}
Enlève les moins bons éléments de la population

\end{fulllineitems}

\index{fitness() (méthode ag\_optimizer.AGOptimizer)}

\begin{fulllineitems}
\phantomsection\label{\detokenize{index:ag_optimizer.AGOptimizer.fitness}}\pysiglinewithargsret{\sphinxbfcode{fitness}}{\emph{vector}}{}
Fitness de l’individu : score total sur nb\_games\_played parties

\end{fulllineitems}

\index{generateNewOffspring() (méthode ag\_optimizer.AGOptimizer)}

\begin{fulllineitems}
\phantomsection\label{\detokenize{index:ag_optimizer.AGOptimizer.generateNewOffspring}}\pysiglinewithargsret{\sphinxbfcode{generateNewOffspring}}{}{}
Renvoie la nouvelle génération

\end{fulllineitems}

\index{initPopulation() (méthode ag\_optimizer.AGOptimizer)}

\begin{fulllineitems}
\phantomsection\label{\detokenize{index:ag_optimizer.AGOptimizer.initPopulation}}\pysiglinewithargsret{\sphinxbfcode{initPopulation}}{}{}
Initialise la population

\end{fulllineitems}

\index{keepOnlyElite() (méthode ag\_optimizer.AGOptimizer)}

\begin{fulllineitems}
\phantomsection\label{\detokenize{index:ag_optimizer.AGOptimizer.keepOnlyElite}}\pysiglinewithargsret{\sphinxbfcode{keepOnlyElite}}{}{}
Garde les meilleurs éléments de la génération précédente

\end{fulllineitems}

\index{makeNewGeneration() (méthode ag\_optimizer.AGOptimizer)}

\begin{fulllineitems}
\phantomsection\label{\detokenize{index:ag_optimizer.AGOptimizer.makeNewGeneration}}\pysiglinewithargsret{\sphinxbfcode{makeNewGeneration}}{}{}
Crée une nouvelle génération

\end{fulllineitems}

\index{mutate() (méthode ag\_optimizer.AGOptimizer)}

\begin{fulllineitems}
\phantomsection\label{\detokenize{index:ag_optimizer.AGOptimizer.mutate}}\pysiglinewithargsret{\sphinxbfcode{mutate}}{\emph{individu}}{}
Mute un individu

\end{fulllineitems}

\index{mutateBinVector() (méthode ag\_optimizer.AGOptimizer)}

\begin{fulllineitems}
\phantomsection\label{\detokenize{index:ag_optimizer.AGOptimizer.mutateBinVector}}\pysiglinewithargsret{\sphinxbfcode{mutateBinVector}}{\emph{bin\_vector}}{}
Mute un vecteur

\end{fulllineitems}

\index{mutateFloatVector() (méthode ag\_optimizer.AGOptimizer)}

\begin{fulllineitems}
\phantomsection\label{\detokenize{index:ag_optimizer.AGOptimizer.mutateFloatVector}}\pysiglinewithargsret{\sphinxbfcode{mutateFloatVector}}{\emph{vector}}{}
Mute un individu (son vecteur)

\end{fulllineitems}

\index{plotStats() (méthode ag\_optimizer.AGOptimizer)}

\begin{fulllineitems}
\phantomsection\label{\detokenize{index:ag_optimizer.AGOptimizer.plotStats}}\pysiglinewithargsret{\sphinxbfcode{plotStats}}{}{}
Courbes de statistiques

\end{fulllineitems}

\index{process() (méthode ag\_optimizer.AGOptimizer)}

\begin{fulllineitems}
\phantomsection\label{\detokenize{index:ag_optimizer.AGOptimizer.process}}\pysiglinewithargsret{\sphinxbfcode{process}}{}{}
Boucle principale de l’optimisation

\end{fulllineitems}

\index{randomBinaryList() (méthode ag\_optimizer.AGOptimizer)}

\begin{fulllineitems}
\phantomsection\label{\detokenize{index:ag_optimizer.AGOptimizer.randomBinaryList}}\pysiglinewithargsret{\sphinxbfcode{randomBinaryList}}{}{}
Renvoie une liste de self.nb\_bits chiffres binaires aléatoires

\end{fulllineitems}

\index{randomBinaryVector() (méthode ag\_optimizer.AGOptimizer)}

\begin{fulllineitems}
\phantomsection\label{\detokenize{index:ag_optimizer.AGOptimizer.randomBinaryVector}}\pysiglinewithargsret{\sphinxbfcode{randomBinaryVector}}{}{}
Renvoie un vecteur binaire aléatoire

\end{fulllineitems}

\index{randomFloatVector() (méthode ag\_optimizer.AGOptimizer)}

\begin{fulllineitems}
\phantomsection\label{\detokenize{index:ag_optimizer.AGOptimizer.randomFloatVector}}\pysiglinewithargsret{\sphinxbfcode{randomFloatVector}}{}{}
Renvoie un vecteur aléatoire normé

\end{fulllineitems}

\index{scoreOnOneGame() (méthode ag\_optimizer.AGOptimizer)}

\begin{fulllineitems}
\phantomsection\label{\detokenize{index:ag_optimizer.AGOptimizer.scoreOnOneGame}}\pysiglinewithargsret{\sphinxbfcode{scoreOnOneGame}}{\emph{vector}}{}
Score sur une partie

\end{fulllineitems}

\index{sortPopulationDescending() (méthode ag\_optimizer.AGOptimizer)}

\begin{fulllineitems}
\phantomsection\label{\detokenize{index:ag_optimizer.AGOptimizer.sortPopulationDescending}}\pysiglinewithargsret{\sphinxbfcode{sortPopulationDescending}}{}{}
Trie la population par ordre décroissant de scores

\end{fulllineitems}

\index{stringOfParameters() (méthode ag\_optimizer.AGOptimizer)}

\begin{fulllineitems}
\phantomsection\label{\detokenize{index:ag_optimizer.AGOptimizer.stringOfParameters}}\pysiglinewithargsret{\sphinxbfcode{stringOfParameters}}{}{}
Renvoie la chaine des paramètres de l’algorithme génétique

\end{fulllineitems}

\index{tournamentSelection() (méthode ag\_optimizer.AGOptimizer)}

\begin{fulllineitems}
\phantomsection\label{\detokenize{index:ag_optimizer.AGOptimizer.tournamentSelection}}\pysiglinewithargsret{\sphinxbfcode{tournamentSelection}}{}{}
Sélection par tournoi

\end{fulllineitems}

\index{updateBinaryIndivdu() (méthode ag\_optimizer.AGOptimizer)}

\begin{fulllineitems}
\phantomsection\label{\detokenize{index:ag_optimizer.AGOptimizer.updateBinaryIndivdu}}\pysiglinewithargsret{\sphinxbfcode{updateBinaryIndivdu}}{\emph{individu}}{}
Met à jour les paramètres d’un individu à partir de son vecteur binaire

\end{fulllineitems}

\index{updateScore() (méthode ag\_optimizer.AGOptimizer)}

\begin{fulllineitems}
\phantomsection\label{\detokenize{index:ag_optimizer.AGOptimizer.updateScore}}\pysiglinewithargsret{\sphinxbfcode{updateScore}}{\emph{individu}}{}
Met à jour le score de l’individu

\end{fulllineitems}

\index{updateStats() (méthode ag\_optimizer.AGOptimizer)}

\begin{fulllineitems}
\phantomsection\label{\detokenize{index:ag_optimizer.AGOptimizer.updateStats}}\pysiglinewithargsret{\sphinxbfcode{updateStats}}{}{}
Met à jours les statistiques

\end{fulllineitems}

\index{wheelSelection() (méthode ag\_optimizer.AGOptimizer)}

\begin{fulllineitems}
\phantomsection\label{\detokenize{index:ag_optimizer.AGOptimizer.wheelSelection}}\pysiglinewithargsret{\sphinxbfcode{wheelSelection}}{}{}
Sélection d’un individu avec une roulette

\end{fulllineitems}


\end{fulllineitems}

\index{binToFloat() (dans le module ag\_optimizer)}

\begin{fulllineitems}
\phantomsection\label{\detokenize{index:ag_optimizer.binToFloat}}\pysiglinewithargsret{\sphinxcode{ag\_optimizer.}\sphinxbfcode{binToFloat}}{\emph{bits}}{}
Renvoie la représentation entre 0 et 1 d’une liste binaire 
Le chiffre des unités étant considéré comme le 1er élément de la liste
(ça n’a aucune importance vu qu’on va partir de listes aléatoires)

\end{fulllineitems}

\index{binVectorToFloat() (dans le module ag\_optimizer)}

\begin{fulllineitems}
\phantomsection\label{\detokenize{index:ag_optimizer.binVectorToFloat}}\pysiglinewithargsret{\sphinxcode{ag\_optimizer.}\sphinxbfcode{binVectorToFloat}}{\emph{bin\_vector}}{}
Convertit un vecteur de listes binaires en vecteur de float

\end{fulllineitems}

\index{getrandbits() (dans le module ag\_optimizer)}

\begin{fulllineitems}
\phantomsection\label{\detokenize{index:ag_optimizer.getrandbits}}\pysiglinewithargsret{\sphinxcode{ag\_optimizer.}\sphinxbfcode{getrandbits}}{\emph{k}}{{ $\rightarrow$ x.  Generates a long int with k random bits.}}
\end{fulllineitems}

\index{linearCombination() (dans le module ag\_optimizer)}

\begin{fulllineitems}
\phantomsection\label{\detokenize{index:ag_optimizer.linearCombination}}\pysiglinewithargsret{\sphinxcode{ag\_optimizer.}\sphinxbfcode{linearCombination}}{\emph{a1}, \emph{vector1}, \emph{a2}, \emph{vector2}}{}
Renvoie la combinaison linéaire de deux vecteurs

\end{fulllineitems}

\index{normalize() (dans le module ag\_optimizer)}

\begin{fulllineitems}
\phantomsection\label{\detokenize{index:ag_optimizer.normalize}}\pysiglinewithargsret{\sphinxcode{ag\_optimizer.}\sphinxbfcode{normalize}}{\emph{vector}}{}
Normalise un vecteur

\end{fulllineitems}

\index{random() (dans le module ag\_optimizer)}

\begin{fulllineitems}
\phantomsection\label{\detokenize{index:ag_optimizer.random}}\pysiglinewithargsret{\sphinxcode{ag\_optimizer.}\sphinxbfcode{random}}{}{{ $\rightarrow$ x in the interval {[}0, 1).}}
\end{fulllineitems}



\chapter{Reinforcement learning}
\label{\detokenize{index:reinforcement-learning}}

\section{Module tetris\_RLenv.py}
\label{\detokenize{index:module-tetris_RLenv}}\label{\detokenize{index:module-tetris-rlenv-py}}\index{tetris\_RLenv (module)}\index{TetrisEnv (classe dans tetris\_RLenv)}

\begin{fulllineitems}
\phantomsection\label{\detokenize{index:tetris_RLenv.TetrisEnv}}\pysiglinewithargsret{\sphinxbfcode{class }\sphinxcode{tetris\_RLenv.}\sphinxbfcode{TetrisEnv}}{\emph{width=10, height=22, max\_blocks=0, base\_blocks\_bag={[}\textless{}tetramino.Tetramino instance\textgreater{}, \textless{}tetramino.Tetramino instance\textgreater{}, \textless{}tetramino.Tetramino instance\textgreater{}, \textless{}tetramino.Tetramino instance\textgreater{}, \textless{}tetramino.Tetramino instance\textgreater{}, \textless{}tetramino.Tetramino instance\textgreater{}, \textless{}tetramino.Tetramino instance\textgreater{}{]}, random\_generator\_seed=None, agent\_name='', agent\_description=''}}{}
Environnement à la OpenAI Gym pour implémenter le reinforcement learning
\index{getState() (méthode tetris\_RLenv.TetrisEnv)}

\begin{fulllineitems}
\phantomsection\label{\detokenize{index:tetris_RLenv.TetrisEnv.getState}}\pysiglinewithargsret{\sphinxbfcode{getState}}{}{}
Renvoie l’état de la grille

\end{fulllineitems}

\index{getStateCode() (méthode tetris\_RLenv.TetrisEnv)}

\begin{fulllineitems}
\phantomsection\label{\detokenize{index:tetris_RLenv.TetrisEnv.getStateCode}}\pysiglinewithargsret{\sphinxbfcode{getStateCode}}{}{}
Renoie de code entier de l’état de la grille

\end{fulllineitems}

\index{render() (méthode tetris\_RLenv.TetrisEnv)}

\begin{fulllineitems}
\phantomsection\label{\detokenize{index:tetris_RLenv.TetrisEnv.render}}\pysiglinewithargsret{\sphinxbfcode{render}}{}{}
Affiche le jeu

\end{fulllineitems}

\index{reset() (méthode tetris\_RLenv.TetrisEnv)}

\begin{fulllineitems}
\phantomsection\label{\detokenize{index:tetris_RLenv.TetrisEnv.reset}}\pysiglinewithargsret{\sphinxbfcode{reset}}{}{}
Réinitialise l’environnement

\end{fulllineitems}

\index{sampleAction() (méthode tetris\_RLenv.TetrisEnv)}

\begin{fulllineitems}
\phantomsection\label{\detokenize{index:tetris_RLenv.TetrisEnv.sampleAction}}\pysiglinewithargsret{\sphinxbfcode{sampleAction}}{}{}
Renvoie une action aléatoire

\end{fulllineitems}

\index{step() (méthode tetris\_RLenv.TetrisEnv)}

\begin{fulllineitems}
\phantomsection\label{\detokenize{index:tetris_RLenv.TetrisEnv.step}}\pysiglinewithargsret{\sphinxbfcode{step}}{\emph{action}}{}
Effectue une action (joue un coup)
Met à jour les informations (done, state)

\end{fulllineitems}


\end{fulllineitems}

\index{getrandbits() (dans le module tetris\_RLenv)}

\begin{fulllineitems}
\phantomsection\label{\detokenize{index:tetris_RLenv.getrandbits}}\pysiglinewithargsret{\sphinxcode{tetris\_RLenv.}\sphinxbfcode{getrandbits}}{\emph{k}}{{ $\rightarrow$ x.  Generates a long int with k random bits.}}
\end{fulllineitems}

\index{random() (dans le module tetris\_RLenv)}

\begin{fulllineitems}
\phantomsection\label{\detokenize{index:tetris_RLenv.random}}\pysiglinewithargsret{\sphinxcode{tetris\_RLenv.}\sphinxbfcode{random}}{}{{ $\rightarrow$ x in the interval {[}0, 1).}}
\end{fulllineitems}



\section{Module qRL\_optimizer.py}
\label{\detokenize{index:module-qrl-optimizer-py}}\label{\detokenize{index:module-qRL_optimizer}}\index{qRL\_optimizer (module)}\index{QRLOptimizer (classe dans qRL\_optimizer)}

\begin{fulllineitems}
\phantomsection\label{\detokenize{index:qRL_optimizer.QRLOptimizer}}\pysiglinewithargsret{\sphinxbfcode{class }\sphinxcode{qRL\_optimizer.}\sphinxbfcode{QRLOptimizer}}{\emph{width=5, height=5, base\_blocks\_bag={[}\textless{}tetramino.Tetramino instance\textgreater{}{]}, max\_episodes=2000, max\_blocks=500, alpha=0.1, gamma=0.9, epsilon\_min=0.01, epsilon\_max=1, epsilon\_delta=0.001}}{}
Optimisation par Q-learning sur une configuration simple
\index{getQIndexes() (méthode qRL\_optimizer.QRLOptimizer)}

\begin{fulllineitems}
\phantomsection\label{\detokenize{index:qRL_optimizer.QRLOptimizer.getQIndexes}}\pysiglinewithargsret{\sphinxbfcode{getQIndexes}}{}{}
Renvoie un tableau dans lequel chaque cellule est le nombre
de fois qu’elle apparaît dans les indices de la Q-Table 
avec des Q-Values toutes non nulles

\end{fulllineitems}

\index{initQValue() (méthode qRL\_optimizer.QRLOptimizer)}

\begin{fulllineitems}
\phantomsection\label{\detokenize{index:qRL_optimizer.QRLOptimizer.initQValue}}\pysiglinewithargsret{\sphinxbfcode{initQValue}}{\emph{s}}{}
Initialise la Q-value de l’état s avec des 0 si s n’est pas encore dans la table

\end{fulllineitems}

\index{learn() (méthode qRL\_optimizer.QRLOptimizer)}

\begin{fulllineitems}
\phantomsection\label{\detokenize{index:qRL_optimizer.QRLOptimizer.learn}}\pysiglinewithargsret{\sphinxbfcode{learn}}{}{}
Lance l’apprentissage

\end{fulllineitems}

\index{play() (méthode qRL\_optimizer.QRLOptimizer)}

\begin{fulllineitems}
\phantomsection\label{\detokenize{index:qRL_optimizer.QRLOptimizer.play}}\pysiglinewithargsret{\sphinxbfcode{play}}{}{}
Joue la partie avec la Q-table crée

\end{fulllineitems}

\index{printQIndexes() (méthode qRL\_optimizer.QRLOptimizer)}

\begin{fulllineitems}
\phantomsection\label{\detokenize{index:qRL_optimizer.QRLOptimizer.printQIndexes}}\pysiglinewithargsret{\sphinxbfcode{printQIndexes}}{}{}
Affiche le nombre de fois que chaque cellule apparaît dans
les indices de la Q-Table en nuances de gris

\end{fulllineitems}

\index{reinit() (méthode qRL\_optimizer.QRLOptimizer)}

\begin{fulllineitems}
\phantomsection\label{\detokenize{index:qRL_optimizer.QRLOptimizer.reinit}}\pysiglinewithargsret{\sphinxbfcode{reinit}}{}{}
Réinitialise l’environnement

\end{fulllineitems}

\index{update() (méthode qRL\_optimizer.QRLOptimizer)}

\begin{fulllineitems}
\phantomsection\label{\detokenize{index:qRL_optimizer.QRLOptimizer.update}}\pysiglinewithargsret{\sphinxbfcode{update}}{\emph{s}, \emph{a}}{}
Met à jour la Q-table de l’état s sur une action a

\end{fulllineitems}


\end{fulllineitems}

\index{argmax() (dans le module qRL\_optimizer)}

\begin{fulllineitems}
\phantomsection\label{\detokenize{index:qRL_optimizer.argmax}}\pysiglinewithargsret{\sphinxcode{qRL\_optimizer.}\sphinxbfcode{argmax}}{\emph{liste}}{}
Renvoie l’indice de la valeur max de liste
ou un indice aléatoire si la liste ne contient que des 0

\end{fulllineitems}

\index{getrandbits() (dans le module qRL\_optimizer)}

\begin{fulllineitems}
\phantomsection\label{\detokenize{index:qRL_optimizer.getrandbits}}\pysiglinewithargsret{\sphinxcode{qRL\_optimizer.}\sphinxbfcode{getrandbits}}{\emph{k}}{{ $\rightarrow$ x.  Generates a long int with k random bits.}}
\end{fulllineitems}

\index{random() (dans le module qRL\_optimizer)}

\begin{fulllineitems}
\phantomsection\label{\detokenize{index:qRL_optimizer.random}}\pysiglinewithargsret{\sphinxcode{qRL\_optimizer.}\sphinxbfcode{random}}{}{{ $\rightarrow$ x in the interval {[}0, 1).}}
\end{fulllineitems}

\index{total\_size() (dans le module qRL\_optimizer)}

\begin{fulllineitems}
\phantomsection\label{\detokenize{index:qRL_optimizer.total_size}}\pysiglinewithargsret{\sphinxcode{qRL\_optimizer.}\sphinxbfcode{total\_size}}{\emph{o}}{}
Renvoie la taille totale d’un objet en mémoire
Code récupéré sur : \sphinxurl{https://code.activestate.com/recipes/577504/} 
et adapté au projet

\end{fulllineitems}



\chapter{Divers}
\label{\detokenize{index:divers}}

\section{Module stats.py}
\label{\detokenize{index:module-stats-py}}\label{\detokenize{index:module-stats}}\index{stats (module)}\index{Stats (classe dans stats)}

\begin{fulllineitems}
\phantomsection\label{\detokenize{index:stats.Stats}}\pysiglinewithargsret{\sphinxbfcode{class }\sphinxcode{stats.}\sphinxbfcode{Stats}}{\emph{data=None}, \emph{filename=''}, \emph{mean\_time=0}, \emph{nb\_bars=10}, \emph{title=''}}{}
Représentation statistique des parties
\index{getAllStats() (méthode stats.Stats)}

\begin{fulllineitems}
\phantomsection\label{\detokenize{index:stats.Stats.getAllStats}}\pysiglinewithargsret{\sphinxbfcode{getAllStats}}{}{}
Récupère tous les indicateurs statistiques

\end{fulllineitems}

\index{getEffectif() (méthode stats.Stats)}

\begin{fulllineitems}
\phantomsection\label{\detokenize{index:stats.Stats.getEffectif}}\pysiglinewithargsret{\sphinxbfcode{getEffectif}}{}{}
Renvoie le nombre de données

\end{fulllineitems}

\index{getMaxi() (méthode stats.Stats)}

\begin{fulllineitems}
\phantomsection\label{\detokenize{index:stats.Stats.getMaxi}}\pysiglinewithargsret{\sphinxbfcode{getMaxi}}{}{}
Renvoie le maximum des données

\end{fulllineitems}

\index{getMean() (méthode stats.Stats)}

\begin{fulllineitems}
\phantomsection\label{\detokenize{index:stats.Stats.getMean}}\pysiglinewithargsret{\sphinxbfcode{getMean}}{}{}
Renvoie la moyenne des données

\end{fulllineitems}

\index{getMini() (méthode stats.Stats)}

\begin{fulllineitems}
\phantomsection\label{\detokenize{index:stats.Stats.getMini}}\pysiglinewithargsret{\sphinxbfcode{getMini}}{}{}
Renvoie le minimum des données

\end{fulllineitems}

\index{getQuartiles() (méthode stats.Stats)}

\begin{fulllineitems}
\phantomsection\label{\detokenize{index:stats.Stats.getQuartiles}}\pysiglinewithargsret{\sphinxbfcode{getQuartiles}}{}{}
Renvoie un tuple (Q1, Médiane, Q3)

\end{fulllineitems}

\index{getSigma() (méthode stats.Stats)}

\begin{fulllineitems}
\phantomsection\label{\detokenize{index:stats.Stats.getSigma}}\pysiglinewithargsret{\sphinxbfcode{getSigma}}{}{}
Renvoie l’écart-type des données

\end{fulllineitems}

\index{histogram() (méthode stats.Stats)}

\begin{fulllineitems}
\phantomsection\label{\detokenize{index:stats.Stats.histogram}}\pysiglinewithargsret{\sphinxbfcode{histogram}}{\emph{save=True}}{}
Crée et affiche l’histogramme

\end{fulllineitems}

\index{loadData() (méthode stats.Stats)}

\begin{fulllineitems}
\phantomsection\label{\detokenize{index:stats.Stats.loadData}}\pysiglinewithargsret{\sphinxbfcode{loadData}}{}{}
Charge les données à partir d’un fichier texte

\end{fulllineitems}

\index{saveData() (méthode stats.Stats)}

\begin{fulllineitems}
\phantomsection\label{\detokenize{index:stats.Stats.saveData}}\pysiglinewithargsret{\sphinxbfcode{saveData}}{}{}
Sauvegarde les données dans un fichier texte

\end{fulllineitems}


\end{fulllineitems}



\section{Module textutil.py}
\label{\detokenize{index:module-textutil-py}}\label{\detokenize{index:module-textutil}}\index{textutil (module)}\index{boxed() (dans le module textutil)}

\begin{fulllineitems}
\phantomsection\label{\detokenize{index:textutil.boxed}}\pysiglinewithargsret{\sphinxcode{textutil.}\sphinxbfcode{boxed}}{\emph{text}, \emph{prefix=''}, \emph{window\_width=0}, \emph{window\_height=0}}{}
Affiche chaque ligne de texte précédée d’un préfixe
dans une boîte de largeur window\_width

\end{fulllineitems}

\index{center() (dans le module textutil)}

\begin{fulllineitems}
\phantomsection\label{\detokenize{index:textutil.center}}\pysiglinewithargsret{\sphinxcode{textutil.}\sphinxbfcode{center}}{\emph{string}, \emph{length}}{}
Centre la chaîne string sur la longueur

\end{fulllineitems}

\index{cleanLine() (dans le module textutil)}

\begin{fulllineitems}
\phantomsection\label{\detokenize{index:textutil.cleanLine}}\pysiglinewithargsret{\sphinxcode{textutil.}\sphinxbfcode{cleanLine}}{\emph{string}}{}
Renvoie la chaine sans les caractères spéciaux de couleur

\end{fulllineitems}

\index{dateNow() (dans le module textutil)}

\begin{fulllineitems}
\phantomsection\label{\detokenize{index:textutil.dateNow}}\pysiglinewithargsret{\sphinxcode{textutil.}\sphinxbfcode{dateNow}}{}{}
Renvoie une chaîne avec la date et l’heure courante

\end{fulllineitems}

\index{mergeChains() (dans le module textutil)}

\begin{fulllineitems}
\phantomsection\label{\detokenize{index:textutil.mergeChains}}\pysiglinewithargsret{\sphinxcode{textutil.}\sphinxbfcode{mergeChains}}{\emph{string1}, \emph{string2}}{}
Fusionne deux chaînes cote à cote pour l’affichage

\end{fulllineitems}

\index{textColor() (dans le module textutil)}

\begin{fulllineitems}
\phantomsection\label{\detokenize{index:textutil.textColor}}\pysiglinewithargsret{\sphinxcode{textutil.}\sphinxbfcode{textColor}}{\emph{string}, \emph{bg=15}, \emph{fg=232}}{}
Renvoie une chaîne contenant le texte coloré
avec bg pour la couleur de fond
et fg pour la couleur du texte.
Utilise les codes ANSI/VT100.

\end{fulllineitems}



\renewcommand{\indexname}{Index des modules Python}
\begin{sphinxtheindex}
\def\bigletter#1{{\Large\sffamily#1}\nopagebreak\vspace{1mm}}
\bigletter{a}
\item {\sphinxstyleindexentry{ag\_optimizer}}\sphinxstyleindexpageref{index:\detokenize{module-ag_optimizer}}
\item {\sphinxstyleindexentry{agent}}\sphinxstyleindexpageref{index:\detokenize{module-agent}}
\item {\sphinxstyleindexentry{agent\_evaluation}}\sphinxstyleindexpageref{index:\detokenize{module-agent_evaluation}}
\item {\sphinxstyleindexentry{agent\_filtering}}\sphinxstyleindexpageref{index:\detokenize{module-agent_filtering}}
\item {\sphinxstyleindexentry{agent\_human}}\sphinxstyleindexpageref{index:\detokenize{module-agent_human}}
\item {\sphinxstyleindexentry{agent\_random1}}\sphinxstyleindexpageref{index:\detokenize{module-agent_random1}}
\item {\sphinxstyleindexentry{agent\_random2}}\sphinxstyleindexpageref{index:\detokenize{module-agent_random2}}
\indexspace
\bigletter{b}
\item {\sphinxstyleindexentry{board}}\sphinxstyleindexpageref{index:\detokenize{module-board}}
\indexspace
\bigletter{q}
\item {\sphinxstyleindexentry{qRL\_optimizer}}\sphinxstyleindexpageref{index:\detokenize{module-qRL_optimizer}}
\indexspace
\bigletter{s}
\item {\sphinxstyleindexentry{stats}}\sphinxstyleindexpageref{index:\detokenize{module-stats}}
\indexspace
\bigletter{t}
\item {\sphinxstyleindexentry{tetramino}}\sphinxstyleindexpageref{index:\detokenize{module-tetramino}}
\item {\sphinxstyleindexentry{tetris\_engine}}\sphinxstyleindexpageref{index:\detokenize{module-tetris_engine}}
\item {\sphinxstyleindexentry{tetris\_RLenv}}\sphinxstyleindexpageref{index:\detokenize{module-tetris_RLenv}}
\item {\sphinxstyleindexentry{textutil}}\sphinxstyleindexpageref{index:\detokenize{module-textutil}}
\end{sphinxtheindex}

\renewcommand{\indexname}{Index}
\printindex
\end{document}